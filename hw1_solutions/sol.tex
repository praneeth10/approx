\documentclass{article}
% \usepackage{kpfonts}
\usepackage{fourier}
\usepackage{amsthm}
\usepackage{amsmath}
\usepackage{amsfonts}
\usepackage{setspace}
\usepackage{listing}
\onehalfspacing
\title{Assignment1}
\author{Praneeth Kacham \\ \texttt{2015CS10600}}
\date{}
\newcommand{\set}[1]{\{#1\}}
\usepackage[margin=2cm]{geometry}
\begin{document}
\maketitle
\section{Question-1}
Optimal has atmost $l$ edge-disjoint paths of length greater than length $l$. Otherwise, the graph will have $> l^2 \geq m$ edges which is a contradiction. If the greedy algorithm gives $k$ edge disjoint paths of length less than $l$, then each of the paths in optimum of length less than $l$ should intersect edges in one of these $k$ paths, otherwise greedy would have picked the path in the optimum which doesn't intersect any of these paths, and no two paths in optimum can intersect the same edge of the $k$-paths given by the greedy algorithm. As the total number of edges in the optimum is less than $kl$, we can only have atmost $kl$ disjoint paths of length less than $l$ in the optimum. So, the no. of disjoint paths in optimum is $\leq l + kl$. But $k$ is greater than 1 as $l > diameter(G)$. Hence, greedy can choose the shortest path between any of the pairs(The length will be less than $l$) which gives $k \geq 1$. So, $OPT \leq l + kl \leq kl + kl \leq 2kl \Rightarrow l \geq (1/2k)OPT$. Hence, the algorithm is a $\Omega(1/k)$-optimal algorithm.

\section{Question-2}
We are given $m$ machines and $n$ jobs each with $p_j \geq OPT/3$. Let the jobs be ordered in a way such that $p_1 \geq p_2 \geq \ldots \geq p_n$. And $k$ be the largest integer such that $p_k > 2OPT/3$. So, in the optimum solution, we have that machines which are allotted jobs 1 through $k$ are given only one job and all other machines are allotted atmost $2$ jobs. So, the total number of jobs is atmost $2n-k$. Now, the greedy algorithm, allots first $k$ jobs to $k$ machines as do the optimal solution. We claim that the greedy algorithm allots the remaining jobs to machines in the following way:
\begin{itemize}
\item if jobs $i$ and $j$ are allotted to $l$th machine then, $i+j = 2n+1$.
\end{itemize}
\begin{proof}
The greedy algorithm allots the next $n-k$ jobs to the remaining $n-k$ machines one after another.  Now the machine with the lightest load is $n$th machine. So, greedy algorithm allots the $n+1$th job to $n$th machine and load on $nth$ machine is $\geq OPT/3+OPT/3 = 2OPT/3$. But the load on the machines $k+1\ldots n-1 \leq 2OPT/3$ as the jobs that are allotted to the machines are of size $\leq 2OPT/3$. Hence, $n+2$th job is allotted to $n-1$th machine. Now the load on $n-1$th machine is $\geq 2OPT/3$. Hence, the next job is assigned to $n-2$th machine and so on. We have atmost $2n-k$ jobs and hence in the worst case, the last job is assigned to $n-k+1$th machine. 

\noindent
Claim: There is an optimum solution which assigns the jobs as the greedy algorithm. 

Let $O$ be an optimal solution. If it is same as greedy, we are done. Otherwise there exists two machines to which jobs $(i,j)$ and $(i',j')$ are assigned such that $i < i'$ and $j < j'$ (We can assume WLOG that $i < j$ and $i' < j'$). As O is optimal, $p_i + p_j < OPT$ and $p_i' + p_j' < OPT$. But $j < j' \Rightarrow p_i + p_{j'} \leq p_i + p_j \leq OPT$ and $i < i' \Rightarrow p_j + p_{i'} \leq p_j + p_i \leq OPT$. So, the schedule O' which assigns the jobs $(i,j')$ and $(i',j)$ to the machines will have makespan atmost makespan(O). But $O$ is optimal. So, makespan(O') = OPT. By continuing this we arrive at the solution given by the greedy algorithm.
\end{proof}


\section{Question-3}
Let $V$ be the set of vertices and $S_1$ and $S_2$ be the parition of $V$. Pick an arbitrary vertex and put it into $S_1$. Now iterate over the remaining vertices and one after another put them into $S_1$ or
$S_2$ depending on which of the configurations gives greater cut size based on the vertices added to $S_1$ and $S_2$ till that point. This is a 1/2-approximation. 
Notation: $v_i$ be the vertex picked in $i$th iteration, $S_1(i), S_2(i)$ be the sets $S_1$ and $S_2$ at the end of the $i$th iteration, $V(i) = S_1(i) \cup S_2(i)$ and $G(i)$ be the graph induced by the vertices $V(i)$.

\noindent
\textbf{Claim:} Partition $(S_1(i),S_2(i))$ is a 1/2-approximation for $G(i)$. 
\begin{proof}
    Proof by induction on $i$.

    True for $i = 1$ as the optimal partition has weight $0$.
    
    Assume that the claim is true for all $k < i$.
    
    Without loss of generality, assume that $v_i \in S_1$. Let $(A,V(i)-A)$ be the optimal partition for $G(i)$ and $v(i) \in A$. We have $OPT(G(i)) = wt((A,V(i)-A)) = wt((A-v(i),V(i)-A))$ + contribution of $v_i$. 
    But $wt((A-v(i),V(i)-A)) \leq OPT(G(i-1))$ and contribution of $v_i \leq{}$ weight of edges incident on $v_i$ in $G(i)$. So,
    \begin{equation}
        OPT(G(i)) \leq OPT(G(i-1)) + \text{weight of edges incident on }v_i
    \end{equation}
        By the induction hypothesis we have $ wt(S_1(i)-v_i, S_2(i)) = wt(S_1(i-1),S_2(i-1)) \geq  OPT(G(i-1))/2 \geq wt((A-v(i),V(i)-A))/2$. We also have $wt(v_i,S_2(i)) \geq $ (weight of edges incident on $v_i$)/2 as $v_i$ is added to $S_1$ by the greedy algorithm. So, $wt(S_1(i),S_2(i)) = wt(S_1(i)-v_i,S_2(i)) + \text{contr. of } v_i \geq OPT(G(i-1))/2 + (\text{wt. of edges incident on } v_i)/2 \geq OPT(G(i))/2$.
\end{proof}
Thus the partition $(S_1(n),S_2(n))$ is a 1/2-approximation for the graph $G(n) = G$.

The generalized algorithm for k partitions is as follows. Start with $S_1,S_2,\ldots,S_k = \phi$. Iterate over the vertices $v_i$. Put the vertex $v_i$ into the set $S_j$ such that
\begin{equation}
    j = \text{argmax}_j wt(S_1,\ldots,S_{j-1},S_j \cup v_i, S_{j+1},\ldots,S_k)
\end{equation}
So,
\begin{align}
    \text{contibution of $v_i$ at the end of ith iteration} &= \max_{j}wt(S_1,S_2,\ldots,S_{j-1},v_i,S_{j+1},\ldots,S_k) \\
    &\geq \frac{1}{k}\sum_{j}wt(S_1,S_2,\ldots,S_{j-1},v_i,S_{j+1},\ldots,S_k)\\
    &\geq \frac{1}{k}(k-1)\text{wt. of edges incident on }v_i
\end{align}
So, given that $wt(S_1(i-1),S_2(i-1)\ldots,S_k(i-1))$ is $\geq (1-1/k)OPT(G(i-1))$ we have $wt(S_1(i),\ldots,S_j(i),\ldots,S_k(i)) = wt(S_1(i-1),S_2(i-1),\ldots,S_j(i-1)\cup v_i,\ldots,S_k(i-1))$ $ = wt(S_1,S_2,\ldots,S_j,\ldots,S_k)$ + contr. of $v_i \geq (1-1/k)OPT(G(i-1)) + (1-1/k)$wt. of edges incident on $v_i$ $\geq (1-1/k)OPT(G(i))$.
Hence $(S_1(n),S_2(n),\ldots,S_k(n))$ is $(1-1/k)$ approximate partition of $G(n) = G$.
\section{Question-4}
Let $S = \set{1,2,3,\ldots,k}$. And each of them have a coverage requirement of $\alpha_i \in \mathbb{Z}^+$. Let $\sum_{i}\alpha_i = n$. The greedy algorithm is as follows:
\begin{enumerate}
    \item $\beta_i = \alpha_i$ forall $i \in S$ and $r_i = 0\ \forall S_i$.
    \item if $\exists i\ \beta_i > 0$, continue else exit
    \item find $j$ such that $\frac{w(S_j)}{|S_j \cap \set{i|\beta_i > 0}|}$ is minimum.
    \item $r_j := r_j + 1$ and $\beta_i := \beta_i - 1\ \forall i \in S_j$
    \item Go to step 2
\end{enumerate}

The above algorithm is a $\ln n$-approximate algorithm. Let $n_i = \sum_j \beta_j$ before $i$th iteration. So, $n_1 = n$ and $n_{i} - n_{i+1}$ is the number of elements covered in $i$th iteration. Let $S_k$ be the set chosen in $i$th iteration. Optimum solution can cover $n_i$ elements with cost $OPT$. So, there should be a set in OPT which can cover some elements with average cost less than $OPT/n_i$. But average cost of $S_k$ should be even less. Hence,
\begin{equation}
    w(S_k) \leq OPT * (n_{i} - n_{i+1})/n_i
\end{equation}
Summing the above inequality over all the iterations, we have
\begin{equation}
    \sum_{j}r_jwt(S_j) \leq OPT*H_n
\end{equation}
So, this algorithm is a $\ln n$-approximation.
\section{Question-5}
Consider three sets of vertices $\{r\}$, $\{v_1,v_2,\ldots,v_m\}$ one for each of the sets $S_i$ and $\{u_1,u_2,\ldots,u_n\}$ one for each of the elements of set $S$. Construct a graph G with vertices as the union of $\{r\}$, $\{v_1,v_2,\ldots,v_m\}$ and $\{u_1,u_2,\ldots,u_n\}$ with the edges as follows, $r \rightarrow v_i$ for all $i=1..n$ with cost of edge $r \rightarrow v_i = wt(S_i)$ and the edges $v_i \rightarrow u_j \ \forall\ i,j\ such\ that\ j\in S_i$ of cost 0. Set the vertices $r,\{u_1,u_2,\ldots,u_n\}$ as required and $r$ as the special vertex and the vertices $\{v_1,v_2,\ldots,v_m\}$ as steiner vertices. Any tree with root $r$ and containing the vertex $u_j$ should have an edge to $v_k$ such that there is an edge. 

\noindent
Let $T = (V',E')$ is a steiner tree. Using this we shall construct a set-cover. Consider the set-cover $ \mathcal{S} = \{S_j| r \rightarrow v_j \in E'\}$.

\noindent
Claim: $\mathcal{S}$ is a vertex cover and $cost(\mathcal{S}) = wt(T)$.
\begin{proof}
Consider an element $e_i \in S$. There is a corresponding vertex
$u_i$ in the vertex set of G and $u_i$ is a required vertex. So, there exists a vertex $v_k$ such that $r \rightarrow v_k$ and $v_k \rightarrow u_i$ which implies $S_k \in \mathcal{S}$  and $e_i \in S_k$. So, $e_i \in \cup_{S_j \in \mathcal{S}}S_j$. As $e_i$ is an arbitrary element of $\mathcal{S}$, we have $\mathcal{S}$ is a set cover. $cost(\mathcal{S}) = \sum_{S_j \in \mathcal{S}}wt(S_j) = \sum_{v_j: r \rightarrow v_j \in E'}wt(r \rightarrow E') = wt(T)$. So, cost of the set-cover $\mathcal{S}$ is $wt(T)$.
\end{proof}

Let $\mathcal{S}$ be a vertex cover. We shall construct a steiner tree $T$. Consider the graph $T = (V,E)$ with $V$ as union of $\{r\}$,$\{v_j | S_j \in \mathcal{S}\}$ and $\{u_1,u_2,\ldots,u_n\}$ and edges as union of $\{r \rightarrow v_j | v_j \in V\}$ and $\{v_j \rightarrow u_k| e_k \in S_j\}$. Remove edges if there are morethan one edges incoming to a vertex of form $u_k$. Given that $\mathcal{S}$ is a setcover, we can see that the tree $T$ is a steiner tree and cost of tree $T$ is equal to weight of set cover $\mathcal{S}$.

From above,
\begin{equation}
    OPT(Steiner Tree) = OPT(Set Cover)
\end{equation}

Hence, given a $O(\log(n))$ approximation algorithm for steiner-tree, we can solve the steiner tree problem corresponding to the setcover problem and give $O(log(n))$ approximation to set-cover problem.


\end{document}

