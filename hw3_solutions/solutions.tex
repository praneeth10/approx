\documentclass{article}
\usepackage[margin=1in]{geometry}
\usepackage{setspace}
\usepackage{amsmath}
\usepackage{amsfonts}
\usepackage{algorithm}
% \usepackage{algorithmic}
\usepackage{caption}
\usepackage{algpseudocode}
\usepackage{amssymb}
\usepackage{float}
\usepackage{amsthm}
\newtheorem{theorem}{Theorem}
\newtheorem{corollary}{Corollary}[theorem]
\newtheorem{lemma}[theorem]{Lemma}
\newtheorem{claim}{Claim}
\newcommand{\sfunction}[1]{\textsf{\textsc{#1}}}
% \onehalfspacing
\renewcommand{\algorithmicrequire}{\textbf{Input:}}
\renewcommand{\algorithmicensure}{\textbf{Output:}}
\newcommand{\R}{\mathbb{R}}
\newcommand{\set}[1]{\{#1\}}
\begin{document}
\section*{Question-1}
Integer-Program
\begin{gather*}
    \min_{x}\quad \sum_{e}c_ex_e \\
    \begin{aligned}
    \textup{s.t.}\quad \sum_{e \in P_i}x_e  &\ge  1\\
                       x_e  &\in  \set{0,1} \\
    \end{aligned}
\end{gather*}
Dual of Relaxed problem
\begin{gather*}
    \max_{y}\quad \sum_{i=1}^ky_i \\
    \begin{aligned}
    \textup{s.t.}\quad \sum_{i : e \in P_i}y_i  &\le  c_e \qquad \forall e \in E\\
                       y_i  &\ge  0 \\
    \end{aligned}
\end{gather*}
\begin{algorithm}
    \caption{Primal-Dual Algorithm for Multi-Cut problem}
    \begin{algorithmic}
        \Procedure{MULTICUT}{$T = (V,E), r \in V, c_e \ge 0\ \forall e, (s_i,t_i)\ i = 1\ldots n$}
            \State $F \leftarrow \varnothing$
            \While{$F$ is not a multi-cut}\Comment{}
            \State $i$  be the index of the unseparated pair $(s_i,t_i)$ having highest $depth(lct(s_i,t_i))$
            \State Increase $y_i$ such till edge $e$ becomes tight
            \State $F \leftarrow F \cup \set{e}$
            \EndWhile
        \EndProcedure
    \end{algorithmic}
\end{algorithm}

In reverse delete step, go through the edges in the reverse order in which they are addded to $F$. Delete an edge $e$ if $F - \set{e}$ is a feasible multi-cut. Return $F$ finally.
\begin{theorem}
    \begin{equation}
        cost(F) = \sum_{e \in F}c_e \le 2*OPT
    \end{equation}
\end{theorem}
\begin{proof}

    Let $y$ be the dual feasible solution given by the algorithm. Hence, $\sum_{i=1}^k y_i \leq OPT$. We also have 
    \begin{flalign*}
        && cost(F) &= \sum_{e \in F}{c_e} \\
        && &= \sum_{e \in F}\sum_{i : e \in P_{i}}y_i && \text{(Since, an edge is added only when tight)}\\
        && &= \sum_{i = 1}^ky_i |F \cap P_{i}|
    \end{flalign*}
\begin{claim}
    $y_i > 0 \Rightarrow |F \cap P_i| \le 2$
\end{claim}
\noindent
\begin{proof}
Let $a \leadsto b$ denote the set of edges in the path from $a$ to $b$. Suppose there is an $i$ such that $y_i > 0$ and $|F \cap P_i| > 2$. Let $u$ be the lowest common ancestor of $s_i$ and $t_i$. Let $e$ be the edge which became tight by increasing $y_i$.(Note: $e$ might have been deleted from $F$ in deletion step). As $|F \cap P_i| > 2$, we can, without loss of generality assume that $|F \cap (s_{i} \leadsto u)| \ge 2$. Let $e_1$, $e_2$ be two edges in $|F \cap (s_{i} \leadsto u)|$ such that $e_1$ is closer to $r$ that $e_2$. Let pair $(s_l,t_l)$ caused the addition of $e_1$ and $(s_{l'},t_{l'})$ caused the addition of $e_2$.
    We claim that our reverse deletion step would have deleted the edge $e_2$ and hence obtain a contradiction. Given that $y_i > 0$, we can conclude that $depth(lct(s_i,t_i)) > depth(lct(s_l,t_l))$ and $depth(lct(s_i,t_i)) > depth(lct(s_{l'},t_{l'}))$. Otherwise, edges $e_1$ or $e_2$ would have been added to $F$ earlier that $e$ and $y_i$ couldn't have been raised. We can also conclude that all the pairs for which $e_2$ is the earliest separator that has been added to $F$ have their lct depth lower that $depth(lct(s_i,t_i))$.Again, otherwise, $e_2$ would have been added before $e$ and hence $y_i$ couldn't have been raised. This implies that all those pairs for which $e_2$ is the earliest separator that has been added to $F$ have $e_1$ in the path between the nodes. It is also easy to see that $e_1$ has been added to $F$ after $e_2$. Thus, in reverse delete step we observe $e_2$ after $e_1$ and hence we will remove $e_2$ from $F$ as $e_1$ separates all the pairs which require $e_2$. Hence having $e_1$ and $e_2$ both in $F$ is a contradiction. Which gives $y_i > 0 \Rightarrow |F \cap P_i| \le 2$.
\end{proof}
From the claim above, we have $y_i|F\cap P_i| \leq 2y_i$. Hence, the $cost(F) = \sum_{i = 1}^ky_i|F \cap P_i| \le \sum_{i=1}^k2y_i \leq 2OPT$.
\end{proof}
\section*{Question-4}
Define $A \cdot X = \sum_{i,j}a_{ij}x_{ij}$.

\noindent
Primal SDP
\begin{gather*}
    \max_{X} \quad \sum_{i < j}w_{ij}(1-x_{ij})/2 \\
    \begin{aligned}
    \textup{s.t.}\quad x_{ii} &= 1 \quad \forall i\\
    X &\succeq 0
    \end{aligned}
\end{gather*}
Dual
\begin{gather*}
    \min_{\gamma} \quad \frac{1}{2}\sum_{i < j}w_{ij} + \frac{1}{4}\sum_{i}\gamma_i\\
    \begin{aligned}
    \textup{s.t.}\quad W + diag(\gamma) \succeq 0\\
    \end{aligned}
\end{gather*}
We have $W$ is a symmetric matrix with $w_{ii} = 0$. To show weak duality, we need to show that given $X \succeq 0$, $x_{ii} = 1\ \forall i$, $W + diag(\gamma) \succeq 0$, we have
\begin{equation*}
    \frac{1}{2}\sum_{i < j}w_{ij}(1-x_{ij}) \leq \frac{1}{2}\sum_{i < j}w_{ij} + \frac{1}{4}\sum_{i}\gamma_i
\end{equation*}
\begin{lemma}
    If $X,Y$ are positive semidefinite matrices, then $X \cdot Y \ge 0$
\end{lemma}
\begin{proof}
    Given matrices $X,Y$ we have $X \cdot Y = \textup{tr}(X^TY)$. As $X,Y$ are p.s.ds, we can write $X = LL^T$ and $Y = MM^T$. Hence, tr($X^TY$) = tr($LL^TMM^T$) = tr($L^TMM^TL$) = tr($L^TM(L^TM)^T$) = $||L^TM||_{F}^2 \ge 0$. Second equality follows from the fact that tr($AB$) = tr($BA$).
\end{proof}
\begin{proof}
    \begin{flalign*}
        && \frac{1}{2}\sum_{i < j}w_{ij}(1-x_{ij}) &\leq \frac{1}{2}\sum_{i < j}w_{ij} + \frac{1}{4}\sum_{i}\gamma_i && \\
        \Leftrightarrow && -\frac{1}{2}\sum_{i < j}w_{ij}x_{ij} &\leq \frac{1}{4}\sum_{i}\gamma_i &&\\
        \Leftrightarrow && -\frac{1}{4}\sum_{i \ne j}w_{ij}x_{ij} &\leq \frac{1}{4}\sum_{i}\gamma_i && (\textup{Since, } x_{ij} = x_{ji}\ \&\ w_{ij} = w_{ji})\\
        \Leftrightarrow && -\frac{1}{4}\sum_{i,j}w_{ij}x_{ij} &\leq \frac{1}{4}\sum_{i}\gamma_i && (\textup{Since, }w_{ii} = 0)\\
        \Leftrightarrow && 0 &\leq \sum_{i,j}w_{ij}x_{ij} + \sum_{i}\gamma_i \\
        \Leftrightarrow && 0 &\leq \sum_{i,j}w_{ij}x_{ij} + \sum_{i}\gamma_i x_{ii} && (\textup{Since, }x_{ii} = 1)\\
        \Leftrightarrow && 0 &\leq (W + diag(\gamma))\cdot X
    \end{flalign*}
    Given that the last inequality is true as both matrices are p.s.ds, we can follow the bi-implications backward and obtain what is required.
\end{proof}
\end{document}
