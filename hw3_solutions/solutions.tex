\documentclass{article}
\usepackage[margin=1in]{geometry}
\usepackage{setspace}
\usepackage{amsmath}
\usepackage{amsfonts}
\usepackage{algorithm}
\usepackage{caption}
\usepackage{algpseudocode}
\usepackage{amssymb}
\usepackage{float}
\usepackage{amsthm}
\usepackage{inconsolata}
% \usepackage{mathpazo}
\usepackage{kpfonts}
\newtheorem{theorem}{Theorem}
\newtheorem{corollary}{Corollary}[theorem]
\newtheorem{lemma}[theorem]{Lemma}
\newtheorem{claim}{Claim}
\newcommand{\sfunction}[1]{\textsf{\textsc{#1}}}
% \onehalfspacing
\renewcommand{\algorithmicrequire}{\textbf{Input:}}
\renewcommand{\algorithmicensure}{\textbf{Output:}}
\newcommand{\R}{\mathbb{R}}
\newcommand{\set}[1]{\{#1\}}
\title{Assignment-3}
\date{}
\author{Praneeth Kacham \\ \texttt{2015CS10600}}
\begin{document}
\maketitle
\section*{Question-1}
Integer-Program
\begin{gather*}
    \min_{x}\quad \sum_{e}c_ex_e \\
    \begin{aligned}
    \textup{s.t.}\quad \sum_{e \in P_i}x_e  &\ge  1\\
                       x_e  &\in  \set{0,1} \\
    \end{aligned}
\end{gather*}
Dual of Relaxed problem
\begin{gather*}
    \max_{y}\quad \sum_{i=1}^ky_i \\
    \begin{aligned}
    \textup{s.t.}\quad \sum_{i : e \in P_i}y_i  &\le  c_e \qquad \forall e \in E\\
                       y_i  &\ge  0 \\
    \end{aligned}
\end{gather*}
\begin{algorithm}
    \caption{Primal-Dual Algorithm for Multi-Cut problem}
    \begin{algorithmic}
        \Procedure{MULTICUT}{$T = (V,E), r \in V, c_e \ge 0\ \forall e, (s_i,t_i)\ i = 1\ldots n$}
            \State $F \leftarrow \varnothing$
            \While{$F$ is not a multi-cut}\Comment{}
            \State $i$  be the index of the unseparated pair $(s_i,t_i)$ having highest $depth(lct(s_i,t_i))$
            \State Increase $y_i$ such till edge $e$ becomes tight
            \State $F \leftarrow F \cup \set{e}$
            \EndWhile
        \EndProcedure
    \end{algorithmic}
\end{algorithm}

In reverse delete step, go through the edges in the reverse order in which they are addded to $F$. Delete an edge $e$ if $F - \set{e}$ is a feasible multi-cut. Return $F$ finally.
\begin{theorem}
    \begin{equation}
        cost(F) = \sum_{e \in F}c_e \le 2*OPT
    \end{equation}
\end{theorem}
\begin{proof}

    Let $y$ be the dual feasible solution given by the algorithm. Hence, $\sum_{i=1}^k y_i \leq OPT$. We also have 
    \begin{flalign*}
        && cost(F) &= \sum_{e \in F}{c_e} \\
        && &= \sum_{e \in F}\sum_{i : e \in P_{i}}y_i && \text{(Since, an edge is added only when tight)}\\
        && &= \sum_{i = 1}^ky_i |F \cap P_{i}|
    \end{flalign*}
\begin{claim}
    $y_i > 0 \Rightarrow |F \cap P_i| \le 2$
\end{claim}
\noindent
\begin{proof}
Let $a \leadsto b$ denote the set of edges in the path from $a$ to $b$. Suppose there is an $i$ such that $y_i > 0$ and $|F \cap P_i| > 2$. Let $u$ be the lowest common ancestor of $s_i$ and $t_i$. Let $e$ be the edge which became tight by increasing $y_i$.(Note: $e$ might have been deleted from $F$ in deletion step). As $|F \cap P_i| > 2$, we can, without loss of generality assume that $|F \cap (s_{i} \leadsto u)| \ge 2$. Let $e_1$, $e_2$ be two edges in $|F \cap (s_{i} \leadsto u)|$ such that $e_1$ is closer to $r$ that $e_2$. Let pair $(s_l,t_l)$ caused the addition of $e_1$ and $(s_{l'},t_{l'})$ caused the addition of $e_2$.
    We claim that our reverse deletion step would have deleted the edge $e_2$ and hence obtain a contradiction. Given that $y_i > 0$, we can conclude that $depth(lct(s_i,t_i)) > depth(lct(s_l,t_l))$ and $depth(lct(s_i,t_i)) > depth(lct(s_{l'},t_{l'}))$. Otherwise, edges $e_1$ or $e_2$ would have been added to $F$ earlier that $e$ and $y_i$ couldn't have been raised. We can also conclude that all the pairs for which $e_2$ is the earliest separator that has been added to $F$ have their lct depth lower that $depth(lct(s_i,t_i))$.Again, otherwise, $e_2$ would have been added before $e$ and hence $y_i$ couldn't have been raised. This implies that all those pairs for which $e_2$ is the earliest separator that has been added to $F$ have $e_1$ in the path between the nodes. It is also easy to see that $e_1$ has been added to $F$ after $e_2$. Thus, in reverse delete step we observe $e_2$ after $e_1$ and hence we will remove $e_2$ from $F$ as $e_1$ separates all the pairs which require $e_2$. Hence having $e_1$ and $e_2$ both in $F$ is a contradiction. Which gives $y_i > 0 \Rightarrow |F \cap P_i| \le 2$.
\end{proof}
From the claim above, we have $y_i|F\cap P_i| \leq 2y_i$. Hence, the $cost(F) = \sum_{i = 1}^ky_i|F \cap P_i| \le \sum_{i=1}^k2y_i \leq 2OPT$.
\end{proof}
\section*{Question-2}
Let $F'$ be the set of all edges initially added by the algorithm. Let $F$ be the set of edges returned by the algorithm which deletes edges in any order i.e., $F$ is a feasible solution for the problem and $\forall e \in F,\  F - \set{e}$ is not feasible. Let $C_i$ be the
connected components in the iteration when $i$th edge is added by the primal-dual algorithm. By, $C_{i}^G \subseteq C_{i}$ denotes the connected components whose
dual variable is increased in the $i$th iteration and define $C_{i}^N = C_{i} - C_{i}^G$. Thus, $C_{0} = \set{\set{s_i},\set{t_i}|(s_i,t_i) \in \mathcal{P}}$, $C_{0}^G = C_0$ and $C_{0}^N = \varnothing$. Define graph $G_i$ as follows: vertex set is given by $V_{i} = \set{v_j | j = 1,\ldots,|C_i|}$ and $E_{i} = \set{(v_k,v_l)|\exists (u,v) \in F, u,v \textup{ in different connected components in }C_i}$.

\begin{claim}
    $G_i$ is a forest for all $i$.
\end{claim}
\begin{proof}
    It is easy to see that the vertex set of $G_i$ along with the edges $F'$ is a forest. As $F \subseteq F'$, we have that $G_i$ is
    a forest.
\end{proof}
\begin{claim}
    All vertices corresponding to the components $C_i^N$ in the graph $G_i$ have degree $\ge 2$.
\end{claim}
\begin{proof}
    Suppose there is a component in $C_i^N$ with degree 1 in the graph $G_i$. Given that the component is in $C_i^N$, we have that the
    connected coponent doesn't separate any pair $(s_i,t_i)$. So, the only edge that is coming into the connected component doesn't connect 
    any pair $(s_i,t_i)$. Hence, this edge can be deleted from $F$ without affecting feasibility. This contradicts the fact that $F - \set{e}$ is unfeasible for all the 
    edges $e \in \set{F}$. Hence, all vertices corresponding to the components $C_i^N$ have a degree $\ge 2$.
\end{proof}
\begin{theorem}
    The algorithm is a 2-approximation to steiner-forest problem.
\end{theorem}
\begin{proof}
    Using the above claims, the proof that this gives 2-approximation goes exactly like the proof of 2-approximateness of the
    reverse-deletion algorithm.
    \end{proof}

\section*{Question-3}
Notation : For $S \subseteq V$, $\delta^+(S) = \set{(i,j) \in A | i \notin S, j \in S}$. Let $\mathcal{S}$ denote the set $\set{S \subseteq V | r \notin S}$.\\
Integer Program
\begin{gather*}
    \min_{x}\quad \sum_{e \in A}c_ex_e \\
    \begin{aligned}
    \textup{s.t.}\quad \sum_{e \in \delta^+(S)}x_e  &\ge  1 \qquad \forall S \in \mathcal{S}\\
                       x_e  &\in  \set{0,1} \\
    \end{aligned}
\end{gather*}
Dual of Relaxation
\begin{gather*}
    \max_{y}\quad \sum_{S \in \mathcal{S}} y_S \\
    \begin{aligned}
    \textup{s.t.}\quad \sum_{S: e \in \delta^+(S)}y_S  &\le  c_e \qquad \forall e \in A\\
                       y_S  &\ge  0 \\
    \end{aligned}
\end{gather*}
\begin{algorithm}
    \caption{Primal-Dual Algorithm for MinCost Branching Problem}
    \begin{algorithmic}
        \Procedure{MIN-BRANCHING}{$G = (V,A), r \in V, c_e \ge 0\ \forall e$}
            \State $F \leftarrow \varnothing$
            \State $\mathcal{C} \leftarrow \set{\set{v} | v \in V-\set{r}}$
            \While{$F$ is not feasible}\Comment{}
            \State Raise the $y_S$ value for all $S \in C$ till an edge $e = (u,v)$ becomes tight
            \State Let $u \in C_1 \in \mathcal{C}$ and $v \in C_2 \in \mathcal{C}$
            \State $\mathcal{C} \leftarrow (\mathcal{C} - \set{C_2}) \cup \set{C_1 \cup C_2}$
            \State $F \leftarrow F \cup \set{e}$
            \EndWhile
        \EndProcedure
    \end{algorithmic}
\end{algorithm}
On the set $F$ returned by the algorithm, do a reverse delete step which goes through the edges in the reverse order they are added to $F$ and delete $e$ from $F$ if 
$F - \set{e}$ is feasible. Let $F'$ be the set finally obtained after the reverse delete step. Let $\mathcal{C}^i$ be the $\mathcal{C}$ in $i$th iteration.
\begin{claim}
    $\sum_{S \in \mathcal{C}^i}|\delta^+(S) \cap F'| \le |\mathcal{C}^i|$
\end{claim}
\begin{proof}
    Suppose the claim doesn't hold. Then $\mathcal{D} = \set{C \in \mathcal{C}^i | |\delta^+(C) \cap F'| \ge 1}$ is non-empty. Let $C$ be a minimal element of $\mathcal{D}$ i.e., $\forall S \subset C : S \notin \mathcal{D}$. It is easy to see that $C$ cannot be set of single vertex.(We would have deleted all but one of the arcs in to the vertex in the reverse delete stage.) From algorithm, we can see that $C$ is made up of two
    components $C_1$ and $C_2$ and one of them is still growing. Let $C_1$ be active and $C_2$ be passive. We also have that there is an edge $e \in F$ going from $C_1$ to $C_2$. From, minimality of $C$, we have $C_1 \notin \mathcal{D}$. We also have that, $C_2$ with the arcs in $F$ in ith iteration has a vertex from which all the vertices can be visited and this vertex is the end-point of the arc $e$. From this we can conclude that $|\delta^+(C) \cap \delta^+(C_1) \cap F'| \ge 1$. Let $e'$ be an arc in $\set{\delta^+(C) \cap \delta^+(C_1) \cap F'}$. Let $e'$ be added in an iteration $j$ in which a
    component $C' \supset C_2$ was growing. But, $C_2$ was already passive which implies that there was another component $C'' \subset C'$ which was also growing. Let $e''$ be the arc which stops the component $C''$ from growing. Now in the reverse delete step, we see $e''$ before seeing $e'$. We claim that $e'$ would have been deleted in the reverse delete step. $r$ could reach all the vertices of $C_2$ using the edge $e''$. Now $e'$ is redundant as all the vertices of $C_2$ could be reached using the vertex which can reach all the vertices of $C_2$. Hence, $e'$ is deleted which contradicts our assumption.
\end{proof}
\begin{proof} of optimality:\\
We have $\sum_{e \in F'}c_e = \sum_{e \in F'}\sum_{S \in \mathcal{S}: e \in \delta^+(S)}y_S = \sum_{S \in \mathcal{S}}y_S|\delta^+(S) \cap F'|$. We claim that $\sum_{S \in \mathcal{S}}y_S|\delta^+(S) \cap F'| \le \sum_{S}y_S$. We prove this by induction on iterations of the algorithm. Initially it is trivially true as both LHS and RHS are true. Assume that the inequality is true before $i$th iteration. In $ith$ iteration LHS increases by $\epsilon\sum_{S \in C^i}|\delta(S) \cap F'|$ and RHS increases by $\epsilon|C^i|$. From the claim above, we have that increase in LHS is less than increase in RHS. Hence, LHS $\le$ RHS, after the end of $i$th iteration. Hence, the algorithm is optimal.
\end{proof}
\section*{Question-4}

Define $A \cdot X = \sum_{i,j}a_{ij}x_{ij}$.

\noindent
Primal SDP
\begin{gather*}
    \max_{X} \quad \sum_{i < j}w_{ij}(1-x_{ij})/2 \\
    \begin{aligned}
    \textup{s.t.}\quad x_{ii} &= 1 \quad \forall i\\
    X &\succeq 0
    \end{aligned}
\end{gather*}
Dual
\begin{gather*}
    \min_{\gamma} \quad \frac{1}{2}\sum_{i < j}w_{ij} + \frac{1}{4}\sum_{i}\gamma_i\\
    \begin{aligned}
    \textup{s.t.}\quad W + diag(\gamma) \succeq 0\\
    \end{aligned}
\end{gather*}
We have $W$ is a symmetric matrix with $w_{ii} = 0$. To show weak duality, we need to show that given $X \succeq 0$, $x_{ii} = 1\ \forall i$, $W + diag(\gamma) \succeq 0$, we have
\begin{equation*}
    \frac{1}{2}\sum_{i < j}w_{ij}(1-x_{ij}) \leq \frac{1}{2}\sum_{i < j}w_{ij} + \frac{1}{4}\sum_{i}\gamma_i
\end{equation*}
\begin{lemma}
    If $X,Y$ are positive semidefinite matrices, then $X \cdot Y \ge 0$
\end{lemma}
\begin{proof}
    Given matrices $X,Y$ we have $X \cdot Y = \textup{tr}(X^TY)$. As $X,Y$ are p.s.ds, we can write $X = LL^T$ and $Y = MM^T$. Hence, tr($X^TY$) = tr($LL^TMM^T$) = tr($L^TMM^TL$) = tr($L^TM(L^TM)^T$) = $||L^TM||_{F}^2 \ge 0$. Second equality follows from the fact that tr($AB$) = tr($BA$).
\end{proof}
\begin{proof}
    \begin{flalign*}
        && \frac{1}{2}\sum_{i < j}w_{ij}(1-x_{ij}) &\leq \frac{1}{2}\sum_{i < j}w_{ij} + \frac{1}{4}\sum_{i}\gamma_i && \\
        \Leftrightarrow && -\frac{1}{2}\sum_{i < j}w_{ij}x_{ij} &\leq \frac{1}{4}\sum_{i}\gamma_i &&\\
        \Leftrightarrow && -\frac{1}{4}\sum_{i \ne j}w_{ij}x_{ij} &\leq \frac{1}{4}\sum_{i}\gamma_i && (\textup{Since, } x_{ij} = x_{ji}\ \&\ w_{ij} = w_{ji})\\
        \Leftrightarrow && -\frac{1}{4}\sum_{i,j}w_{ij}x_{ij} &\leq \frac{1}{4}\sum_{i}\gamma_i && (\textup{Since, }w_{ii} = 0)\\
        \Leftrightarrow && 0 &\leq \sum_{i,j}w_{ij}x_{ij} + \sum_{i}\gamma_i \\
        \Leftrightarrow && 0 &\leq \sum_{i,j}w_{ij}x_{ij} + \sum_{i}\gamma_i x_{ii} && (\textup{Since, }x_{ii} = 1)\\
        \Leftrightarrow && 0 &\leq (W + diag(\gamma))\cdot X
    \end{flalign*}
    Given that the last inequality is true as both matrices are p.s.ds, we can follow the bi-implications backward and obtain what is required.
\end{proof}
\section*{Question-5}
\subsection*{Part-a}
For each vairable $x_i$ in the satisfiability problem we will have a variable $y_i$ which takes the values from the set $\set{-1,1}$. We also have a variable $y_0 \in \set{-1,1}$.
$x_i$ is assigned the truth value TRUE if $y_i\dot y_0 = 1$ and FALSE otherwise. Let the first variable in $ith$ clause be $x_{i_1}$ and
second variable be $x_{i_2}$. For a clause $x_1 \lor x_2$, we have the integer expression $\frac{1}{4}(3 + y_0y_1 + y_0y_2 - y_1y_2)$ and similar expressions for other forms of 2-SAT clauses. Let there be $n$ clauses.
The following integer program models a Max-2SAT problem.

\begin{gather*}
    \max_{y} \quad \sum_{i=1}^n \frac{1}{4}w_i(3 \pm y_0y_{i_1} \pm y_0y_{i_2} \pm y_{i_1}y_{i_2})\\
    \begin{aligned}
    \textup{s.t.}\quad y_j \in \set{-1,1} \ \forall j
    \end{aligned}
\end{gather*}
Sign of $\pm$ depends on the corresponding clause.
\subsection*{Part-b}
The integer program in part-a can be relaxed to get a Semi-definite program by replacing $y_j$ with vector $\vec{y_j}$ and replacing the product $y_jy_k$ with $\langle \vec{y_j}, \vec{y_k} \rangle$ and adding a condition that $||y_j||^2 = 1$.

\noindent
\textbf{Rounding the semi-definite solution:}\\
Let $\vec{y_0}^*, \vec{y_1}^*, \ldots, \vec{y_m}^*$ be the optimal semi-definite solution. Pick a random plane given by $a^Tx = 0$. Variable $y_i$ is
assigned 1 if $a^T\vec{y_i}^* > 0$ and -1 otherwise.

\begin{claim}
    $\forall i,j$, the following hold true
    \begin{flalign*} 
        E[1 + y_iy_j] &\ge 0.878[1 + \langle y_i, y_j \rangle]\\
        E[1 - y_iy_j] &\ge 0.878[1 - \langle y_i,y_j \rangle]
    \end{flalign*}
\end{claim}
\begin{proof}
    We have the following,
    \begin{equation}
    y_iy_j = \begin{cases} +1 \textup{ with probability } 1 - \cos^{-1}(\langle \vec{y_i}^*, \vec{y_j}^*\rangle)/\pi \\ -1 \textup{ with probability } \cos^{-1}(\langle \vec{y_i}^*, \vec{y_j}^* \rangle)/\pi\end{cases}
    \end{equation}
    Hence, $E[1 + y_iy_j] = 2[1 - \cos^{-1}(\langle \vec{y_i}^*, \vec{y_j}^* \rangle)/\pi] = (2/\pi)\cos^{-1}(-\langle \vec{y_i}^*, \vec{y_j}^* \rangle) \ge 0.878(1 + \langle \vec{y_i}^*, \vec{y_j}^* \rangle)$. Similarly, we can show that the other inequality is also true.
\end{proof}
\begin{theorem}
    Rounding as defined is 0.878 approximate solution for MAX-2-SAT.
\end{theorem}
\begin{proof}
Consider a clause $(x_1 \lor x_2)$. The corresponding expression in terms of $y_i's$ is given by $(3 + y_1y_0 + y_2y_0 - y_1y_2)/4$. Then expression
has value $1$ if the clause is satisfied and $0$ otherwise. The expression can also be written as $((1+y_1y_0) + (1 + y_2y_0) + (1 - y_1y_2))/4$. But, from the previous claim, $E[((1+y_1y_0) + (1 + y_2y_0) + (1 - y_1y_2))/4] \ge 0.878 ((1+\langle \vec{y_1}^*,\vec{y_0}^*\rangle) + (1 + \langle \vec{y_2}^*, \vec{y_0}^* \rangle) + (1 - \langle \vec{y_1}^*, \vec{y_2}^* \rangle))/4$. Hence, the
expected value for integer rounding is $ \ge 0.878OPT\_SDP \ge 0.878OPT$.
\end{proof}
\subsection*{Part-c}
We can combine, the 0.75 approx algorithm and MAX-2-SAT algorithm to obtain a better approximation ratio. We have that, for clauses of size greater than 3, 0.789 fraction of them are
true in 0.75-approx algorithm and 0.75 fraction for clauses of size $<$ 3. If,  with $\alpha$ probability 0.75 algorithm is run and $(1-\alpha)$ probability MAX-2-SDP is run, we obtain a $\min\{\alpha 0.789, \alpha 0.75 + (1 - \alpha)0.878\}$ algorithm. This is maximized when $\alpha 0.039 = (1 - \alpha)0.878 \Rightarrow \alpha = 22.5/23.5 = 0.957$ and the best approximation is $\sim$0.755.
\end{document}
